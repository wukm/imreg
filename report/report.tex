\documentclass[10pt,twocolumn]{article}

\usepackage[english]{babel}
\usepackage{blindtext}
\usepackage{amsmath, amsthm, amssymb, amsfonts}
\usepackage{amsxtra, amscd, geometry, graphicx}
\usepackage{endnotes}
\usepackage{cancel}
\usepackage{alltt}
%\usepackage[all,cmtip]{xypic}
\usepackage{mathrsfs}
\usepackage{listings}
%\usepackage{subfigure}
\usepackage{subcaption}
%\usepackage[pdftex]{hyperref}
%\usepackage[dvips,bookmarks,bookmarksopen,backref,colorlinks,linkcolor={blue},citecolor={blue},urlcolor={blue}](hyperref}

% Makes the margin size a little smaller.
\geometry{letterpaper,margin=1.3in}
\usepackage{titlesec}

\titleformat*{\section}{\large\bfseries}
\titleformat*{\subsection}{\normalsize\bfseries}
% Possible font packages. Choose one and comment out the rest.
%\usepackage{times}%
%\usepackage{helvet}%
%\usepackage{palatino}%
%\usepackage{bookman}%

% These are italic.
%\theoremstyle{plain}
%\newtheorem{thm}{Theorem}
%\newtheorem*{thm*}{Theorem}
%\newtheorem{prop}{Proposition}
%\newtheorem*{prop*}{Proposition}
%\newtheorem{conj}{Conjecture}
%\newtheorem*{conj*}{Conjecture}
%\newtheorem{lem}{Lemma}
%  \makeatletter
%  \@addtoreset{lem}{thm}
%  \makeatother 
%\newtheorem*{lem*}{Lemma}
%\newtheorem{cor}{Corollary}
%  \makeatletter
%  \@addtoreset{cor}{thm}
%  \makeatother 
%\newtheorem*{cor*}{Corollary}
%
%%\newtheorem{lem}[thm]{Lemma}
%%\newtheorem{remark}[thm]{Remark}
%%\newtheorem{cor}[thm]{Corollary}
%%\newtheorem{prop}[thm]{Proposition}
%%\newtheorem{conj}[thm]{Conjecture}
%
%% These are normal (i.e. not italic).
%\theoremstyle{definition}
%\newtheorem*{ack*}{Acknowledgements}
%\newtheorem*{app*}{Application}
%\newtheorem*{apps*}{Applications}
%\newtheorem{defn}{Definition}
%\newtheorem*{defn*}{Definition}
%\newtheorem{eg}{Example}
%  \makeatletter
%  \@addtoreset{eg}{thm}
%  \makeatother 
%\newtheorem*{eg*}{Example}
%\newtheorem*{egs*}{Examples}
%\newtheorem{ex}{Exercise}
%\newtheorem*{ex*}{Exercise}
%\newtheorem*{quest*}{Question}
%\newtheorem{rem}{Remark}
%\newtheorem*{rem*}{Remark}
%\newtheorem{rems}{Remarks}
%\newtheorem*{rems*}{Remarks}
%\newtheorem{prob}{Problem}
%\newtheorem*{prob*}{Problem}
%\newtheorem*{soln*}{Solution}
%\newtheorem{soln}{Solution}
%
%
%% New Commands: Common Math Symbols
\providecommand{\R}{\mathbb{R}}%
%\providecommand{\N}{\mathbb{N}}%
%\providecommand{\Z}{{\mathbb{Z}}}%
%\providecommand{\sph}{\mathbb{S}}%
%\providecommand{\Q}{\mathbb{Q}}%
%\providecommand{\C}{{\mathbb{C}}}%
%\providecommand{\F}{\mathbb{F}}%
%\providecommand{\quat}{\mathbb{H}}%
%
%% New Commands: Operators
%\providecommand{\Gal}{\operatorname{Gal}}%
%\providecommand{\GL}{\operatorname{GL}}%
%\providecommand{\card}{\operatorname{card}}%
%\providecommand{\coker}{\operatorname{coker}}%
%\providecommand{\id}{\operatorname{id}}%
%\providecommand{\im}{\operatorname{im}}%
%\providecommand{\diam}{{\rm diam}}%
%\providecommand{\aut}{\operatorname{Aut}}%
%\providecommand{\inn}{\operatorname{Inn}}%
%\providecommand{\out}{{\rm Out}}%
%\providecommand{\End}{{\rm End}}%
%\providecommand{\rad}{{\rm Rad}}%
%\providecommand{\rk}{{\rm rank}}%
%\providecommand{\ord}{{\rm ord}}%
%\providecommand{\tor}{{\rm Tor}}%
%\providecommand{\comp}{{\text{ $\scriptstyle \circ$ }}}%
%\providecommand{\cl}[1]{\overline{#1}}%
%\providecommand{\tr}{{\sf trace}}%
%
%\renewcommand{\tilde}[1]{\widetilde{#1}}%
%\numberwithin{equation}{section}
%
%\renewcommand{\epsilon}{\varepsilon}
%
%\newcommand*\rfrac[2]{{}^{#1}\!/_{#2}}
%
%% This makes the spacing between lines of font a little bigger.  I like the way it looks.
%\newcommand{\spacing}[1]{\renewcommand{\baselinestretch}{#1}\large\normalsize}
%\spacing{1.2}

% END PREAMBLE %%%%%%%%%%%%%%%%%%%%%%%%%
%%%%%%%%%%%%%%%%%%%%%%%%%%%%%%%%%%%%%%%%

\title{Graph Matching in $\R^2$}
\author{Lucas Wukmer}
\date{May 12, 2015}

\begin{document}

\maketitle
\tableofcontents

\section{Abstract}
A energy-maximum method involving Rayleigh quotients is utilized to identify a
given subset of points in the xy-plane which has been randomly transformed
(rigid with scaling) with their corresponding points in the original set. Our
method views these points as a subgraph of a larger attributed (complete)
graph, whose edge attributes are the euclidan distance between these points

\section{Problem Overview \& Notation}
In a simplied version of this problem, we consider a set of $N$ random points
in the xy-plane, and some smaller subset of size $n$:

\begin{subequations}
  \begin{align}
    S_A := \left\{ (x,y) \in \R^2 \ | \  0 \leq x,y \leq 100 \right\} \\
    S_X \subset S_A \\
  N := |S_A|, \quad n := |S_X|
  \end{align}
\end{subequations}

As this choice of notation may suggest, it is often more convenient to view
this as matrices:

\begin{subequations}
  \begin{align}
    %A := \begin{bmatrix}
    %  \vdots  & \vdots & \  & \vdots \\
    %  a_1     & a_2    & \cdots   & a_N \\
    %  \vdots  & \vdots & \  & \vdots
    %\end{bmatrix} \in \R^{2\times N}\\
    A := \begin{bmatrix}
      a_1     & a_2    & \cdots   & a_N
    \end{bmatrix} \in \R^{2\times N}\\
    X := \begin{bmatrix}
      x_1     & x_2    & \cdots   & x_n
    \end{bmatrix} \in \R^{2\times n}\\
    N \geq n
  \end{align}
\end{subequations}

Of course, we consider it ``known'' the correspondence between X and A. That
is, we know the $n$ pairs $(i,j)$ such that $x_i = a_j$.

We then wish to apply a (reversible) affine transformation $T_{k, \theta, v}$
to the subsystem X as follows (given as homogeneous matrices for
simplification):

\begin{equation}
  \begin{aligned}
    \begin{bmatrix}X' \\ 1\end{bmatrix} &:=  T_{k, \theta, v} \begin{bmatrix}X \\ 1\end{bmatrix} \\
    &= \begin{bmatrix}
  k\cos\theta & -k\sin\theta & v_x \\
  k\sin\theta & k\cos\theta & v_y \\
  0             &   0 & 1
\end{bmatrix} \begin{bmatrix}X\\1\end{bmatrix} \\
k \in \R_+, \ \theta &\in [0, 2\pi),  \ v = (v_x, v_y) \in \R^2
\end{aligned}
\end{equation}

Our goal is simply to find a matching between X' and the original A without
knowledge of the transformation $T$ itself. This matching may be thought of as
the matching matrix $M \in \R^{N\times n}$ such that

\begin{equation}
  MA = X
\end{equation}

where\footnote{Yes, I defined three (soon four) systems of notation and somehow managed to abuse them all simulaneously.}
\begin{equation}
  M_{ij} := \delta_{ij} \\
  = \left\{
    \begin{array}{lr}
    1 &: T(a_i) = x_j  \\
    0 &: T(a_i) \notin S_x
  \end{array}
  \right.
\end{equation}

\section{Methods}
We generate a point set $A$ size $N$ and subset $X$ size $n\leq N$ of uniformly
random floats in the range $[0,100]$. and transform each X with a uniform
random $\theta \in [0, 2\pi)$ and $k\in [1,10]$ (our translation vector $v$ is
simply the one that makes $X'$ have a minimum $x$ and $y$ value of 0 each.


We emultate the program introduced by Cour, et al.\cite{citation01} and model
this problem as a graph matching problem.
Specifically, we form two symmetric matrices
$D\in\R^{N \times N} , d\in\R^{n \times n}$ where
\[
  D_{ij} := || a_i - a_j ||_2 ,  \quad d_{ij} := || x_i - x_j ||_2
\]

We view these as edge attributes of the complete graph corresponding to A and its subgraph (which corresponds to X).

The similarity matrix $W \in \R^{Nn \times Nn}$ is then formed as follows:
\[
  W_{ii',jj'} := \frac{1}{1 + \left| \  d_{ij} - D_{i'j'}  \right| }
\]

That is, the $(ii',jj')$ element represents a ``score'' associated with a
matching containing both $(i,i'),(j,j')$. Note the ordering of elements along
either axis corresponds exactly to their order in the vectorized version of the
matching matrix $M$ above. The correct matching matrix $M$ is found as the
maximum of the following quadratic problem:

\[
  M \sim x_M := \mathsf{argmax} \left\{ x^T W x \ | \ Cx \leq b \right\}
\]

with $x_M$ the vectorized version of M. The condition $Cx<b$ refers to a set of
affine conditions that guarantee each row in $M$ has at most one nonzero entry
(i.e. that the matching is one-to-one).


We then calculate the leading eigenvector of the $PWP$, where
$P \in \R^{Nn \times Nn}$ is given by (add this in \dots)

\subsection{Discretization of Matching Matrix}
I normalized each by percents first, so that the largest \emph{relative} element of each column was chosen first.

\section{Results}

We generated 200 random graphs with $N=25 n=4$ and attempted to do this problem with no scaling.
Over 200 trials, the average accuracy (correct number of nodes matched out of $n$) was $77.8\%$.

In the case that the subgraphs were clustered within the graph, the accuracy improved to $81.4\%$

I ran 50 trials on a linear mesh to estimate a scaling factor and was able to the correct scaling factor very closely:
\begin{verbatim}
best energy:  14.081932303
corresp. accuracy:  100.0
estimated scale_factor 1.2828789984
real scale value: 1.277721655652504
\end{verbatim}

Note that the correct energy is simply $n^2=16$ given our similarity matrix.
\begin{figure}[t]
    \centering
    \includegraphics[width=.4\textwidth]{sample}
    \caption{jlsakdsf}
    %\label{fig:stuff}
\end{figure}

\section{Conclusion}

\begin{itemize}

  \item Thank you James.
  \item The devil and Hitler had a baby and its name was matplotlib.pyplot
  \item I need more RAM
\end{itemize}

\begin{thebibliography}{9}
\bibitem{citation01}
  Cour, Timothee, Praveen Srinivasan, and Jianbo Shi.
  \emph{"Balanced graph matching."}
  Advances in Neural Information Processing Systems 19 (2007): 313.
\end{thebibliography}
\end{document}
